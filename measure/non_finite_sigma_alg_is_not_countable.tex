% vim: foldmethod=marker: 
\documentclass[11pt,twoside,a4paper]{article}

%|--- Packages and Configs {{{1
\usepackage[utf8]{inputenc}
\usepackage[T1]{fontenc}    
\usepackage[brazil]{babel}

\usepackage{amsmath}
\usepackage{amsfonts}
\usepackage{amssymb}
\usepackage{amsthm}
\usepackage{mathrsfs}

\usepackage[top=2cm, bottom=2cm, left=2cm, right=2cm]{geometry}
\usepackage{mathtools}
\usepackage{hyperref}
\usepackage{enumerate}

\usepackage{bbm} %% \mathbbm{1} gives you the identity function symbol 1

%%% allows you to insert many figures indexed by (a), (b), ... on a figure environment
\usepackage{float} 
\usepackage[caption = false]{subfig}

\usepackage{tikz}
\usetikzlibrary{snakes} %% produces curly arrows on tikz
\usetikzlibrary{matrix} %% for commutative diagrams
\usetikzlibrary{arrows}

\usepackage{ifthen} %% gives if conditionals when using newcommand

\setcounter{secnumdepth}{0} %%% this will print the section titles on tableofcontens without inserting numbers.
%% Garamond fonts 
%% \usepackage{ebgaramond}
%% \usepackage[ugm]{mathdesign}
% 1}}}

%|--- Author, Title and Date {{{1
\title{Any non-finite sigma algebra is not countable}
\author{Rafael Polli Carneiro}
\date{} 
% 1}}}

%|--- Math operators {{{1
\DeclareMathOperator {\card}{card}
\DeclareMathOperator {\Id}{\mathbbm{1}}
\DeclareMathOperator {\N}{\mathbb{N}}



\DeclareTextFontCommand{\emph}{\bfseries\em} %% redefining \emph{} with bold and italic font
\DeclareMathOperator{\definedAs}{\vcentcolon = }

\DeclareMathOperator{\RMod}{ R-\text{módulo}}
\DeclareMathOperator {\Imagem}{ Im }
% 1}}}

%|--- Theorems {{{1

\theoremstyle{remark}
\newtheorem{example}{Example}[section]

\theoremstyle{definition}
\newtheorem{definition}{Definition}

\theoremstyle{plain}
\newtheorem{proposition}{Proposition}
% 1}}}

%|--- Begin Document {{{1
\begin{document}
\maketitle

Let's prove the statement of the title. But first 
\begin{definition}
    Let $\Omega$ be a non-empty set. Then any collection of subsets of
    $\Omega$, namely $\mathcal{A}$, is a $\sigma$-algebra of $\Omega$ if
    it the satisfies following properties
    \begin{enumerate}[(i)]
        \item 
            $A \in \mathcal{A} \implies A \subseteq \Omega$;

        \item
            $\emptyset \in \Omega$;

        \item
            For any countable subset of $\mathcal{A}$ we have that
            its union is still an element of $\mathcal{A}$. That
            is:
            \begin{equation*}
                \forall \mathcal{F} \subseteq \mathcal{A},
                \;
                \text{if } \mathcal{F} \text{ is countable}
                \implies
                \bigcup \mathcal{F} \in \mathcal{A}.
            \end{equation*}

        \item
            For any $A \in \mathcal{A}$ we have $A^c \in \mathcal{A}$.
    \end{enumerate}
\end{definition}

Now, given the definition of a $\sigma$-algebra, we will prove the 
desired.
\begin{proposition}
    Any non-finite $\sigma$-algebra of a set $\Omega$ is not countable.    
\end{proposition}

\begin{proof}
    Let $\mathcal{A}$ be a non-finite $\sigma$-algebra of subsets of
    $\Omega$ ($\Omega$ non-empty). We start noticing that it must exist
    a subset $A_0 \in \mathcal{A}$ such that
    \begin{equation*}
        A_0 \neq \emptyset
        \quad \text{and} \quad
        A_0 \neq \Omega.
    \end{equation*}
    With that we can break $\Omega$ into two slices $A_0$ and $A_0^c$.
    Here $A_0^c \in \mathcal{A}$ because of the property of a $\sigma$-algebra.
    That is
    \begin{equation*}
        \begin{tikzpicture}[scale=5]
            \draw[|-|] (0,0)--(1,0) node[midway, above] {$A_0$};
            \draw[|-|]   (1,0) -- (2,0) node[midway, above] {$A_0^c$};
        \end{tikzpicture}
    \end{equation*}
    Now, it comes from the hypothesis that $\mathcal{A}$ is not finite
    that either
    \begin{equation*}
        \card \{A \in \mathcal{A}; \; A \subseteq A_0\} < \infty
        \quad \text{or} \quad
        \card \{A \in \mathcal{A}; \; A \subseteq A_0^c\} < \infty.
    \end{equation*}
    This is true due to the fact that for all $A \in \mathcal{A}$
    we have $A = (A\cap A_0) \cup (A\cap A_0^c)$ which means
    \begin{equation*}
        \mathcal{A} = (\mathcal{A}\cap A_0) \bigcup (\mathcal{A}\cap A_0^c),
    \end{equation*}
    where 
    \begin{equation*}
        \mathcal{A}\cap B \definedAs  \{A \cap B;\; A \in \mathcal{B}\}
    \end{equation*}
    and $B \subseteq \Omega$ a set.
    Then, if both $\mathcal{A}\cap A_0, \mathcal{A}\cap A_0^c$ were
    finite  $\mathcal{A}$ would be also finite, which is impossible 
    by hypothese. 

    Let's consider, without loss of generalization, that $\mathcal{A}\cap A_0$
    is not finite. What we will observe is that we can continue this argument
    recursively. This is so beacuse $\mathcal{A}\cap A_0$ is also a $sigma$-algebra,
    which as showed before is not finite. Therefore, we can build by 
    induction a sequence of $\sigma$-algebras 
    \begin{equation*}
        \mathcal{A} \cap A_0 \supseteq \mathcal{A} \cap A_1 \supseteq \mathcal{A} \cap A_2
        \supseteq \cdots \supseteq
        \mathcal{A} \cap A_n \supseteq \mathcal{A} \cap A_{n+1} \cdots
    \end{equation*}
    where for each index $i \in \N \cup \{0\}$ there is a point $x_{i}$
    such that  $x_i \notin A_{i+1}$. In other words, we have the following
    \begin{equation*}
        \begin{tikzpicture}[
            every path/.style={thick}
            ]
            %\draw[-,dashed,red] (8,1) -- (8,-5);
            %\draw[-,dashed,red] (4,1) -- (4,-5);
            %\draw[-,dashed,red] (2,1) -- (2,-5);
            %\draw[-,dashed,red] (1,1) -- (1,-5);

            \draw[|-|] (0,0) -- (16,0) node[near end, above] {$A_0^c$};
            \draw[|-|] (0,0) -- (8,0) node[midway, above]  {$A_0$};
            \fill[color=red] (7,0) circle (0.03cm) node[above]  {$x_0$};

            \draw[|-|] (0,-1) -- (8,-1) node[near end, above] {$A_1^c$};
            \draw[|-|] (0,-1) -- (4,-1) node[midway, above] {$A_1$};
            \fill[color=red] (3,-1) circle (0.03cm) node[above]  {$x_1$};
                            
            \draw[|-|] (0,-2) -- (4,-2) node[near end, above] {$A_2^c$};
            \draw[|-|] (0,-2) -- (2,-2) node[midway, above] {$A_2$};
            \fill[color=red] (1.5,-2) circle (0.03cm) node[above]  {$x_2$};
                            
            \draw[|-|] (0,-3) -- (2,-3) node[near end, above] {$A_3^c$};
            \draw[|-|] (0,-3) -- (1,-3) node[midway, above] {$A_3$};
            \fill[color=red] (0.8,-3) circle (0.03cm) node[below]  {$x_3$};

            \fill (1/2,-4) circle   (0.03cm);
            \fill (1/2,-4.3) circle (0.03cm);
            \fill (1/2,-4.6) circle (0.03cm);

        \end{tikzpicture}
    \end{equation*}
    Notice that this property is merely a consequence of what was showed at the beggining
    and used at the induction.

    Now comes the interesting part. Firstly, we must look at the sequence
    of non-empty measurable sets
    \begin{equation*}
        B_0 = A_0 \setminus A_1 \in \mathcal{A},
        B_1 = A_1 \setminus A_2 \in \mathcal{A},
        \ldots
        B_i = A_i \setminus A_{i+1} \in \mathcal{A},
        \ldots
    \end{equation*}
    which satisfies $B_i \cap B_j  = \emptyset$, for indexes $i \neq j$.
    Now, for convinience we will define the collections
    \begin{equation*}
        \mathcal{F}_i \definedAs \{B_{2i -1},B_{2i -2} \}, \forall i \in \N
    \end{equation*}
    which will look like
    \begin{equation*}
        \mathcal{F}_1 = \{B_1, B_0\},
        \mathcal{F}_2 = \{B_3, B_2\},
        \mathcal{F}_3 = \{B_5, B_4\},
        \ldots
    \end{equation*}
    Then, remember that the set
    \begin{equation*}
        \mathcal{C} \definedAs \{f:\N \to \{0,1\}; \; f \text{ a function} \}
    \end{equation*}
    is not countable 
    (check the Cantor's diagonal argument to prove this fact
    \footnote{
        \url{https://en.wikipedia.org/wiki/Cantor's_diagonal_argument}
    }). 
    and define the sequence of functions
    \begin{align*}
        \xi_i: \{0,1\} & \to \mathcal{F}_i = \{B_{2i -1},B_{2i -2} \}\\
        j & \mapsto 
        \begin{cases}
        B_{2i -2} & \text{if } j = 0;\\
        B_{2i -1} & \text{if } j = 1,
        \end{cases}
    \end{align*}
    with $i \in \N$. Notice that the set $\mathcal{C}$ is the key to
    prove the proposition succesfully.

    Finally, we just need to prove that the following
    function 
    \begin{align*}
        \phi: \mathcal{C} & \to \mathcal{A} \\
        f & \mapsto \bigcup_{i \in \N} \xi_i(f(i)).
    \end{align*}
    is injective. Whenever the injection is proved we conclude
    that
    \begin{equation*}
        \card( \mathcal{C} ) = \card( \Imagem(\phi) ) \leq \card( \mathcal{A} )
    \end{equation*}

    Let $f,g \in \mathcal{C}$ be two
    different functions. That is, we can find a natural
    number $j \in \N$ such that $f(j) \neq g(j)$.
    Without loss of generalization, lets assume that
    $f(j) = 0$ and $g(j) = 1$.  This gives us
    \begin{equation*}
        \xi_j(f(j)) = B_{2j-2},
        \quad \text{and} \quad,
        \xi_j(g(j)) = B_{2j-1}.
    \end{equation*}
    Thus, from the fact that 
    \begin{equation*}
        \forall i\neq j, \; B_i \cap B_j = \emptyset
    \end{equation*}
    we obtain
    \begin{equation*}
        B_{2j-2} \cap \bigcup_{i \in \N} \xi_i(g(i)) = 
        \bigcup_{i \in \N} \left( \xi_i(g(i)) \cap B_{2j-2} \right) = 
        \left(B_{2j-1}\cap B_{2j-2} \right)
        \cup
        \bigcup_{i \in \N \setminus \{j\}} \left( \xi_i(g(i)) \cap B_{2j-2} \right)
        = \emptyset.
    \end{equation*}
    Consequently, $\xi(f) \neq \xi(g)$ and $\xi$ is a injective function.
    Hence, the $\sigma$-algebra is not countable!!!
\end{proof}


\end{document}
% 1}}}


