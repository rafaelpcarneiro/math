% vim: foldmethod=marker: 
\documentclass[11pt,a4paper]{article}
% vim: foldmethod=marker: 

%|--- Packages and Configs {{{1
\usepackage[utf8]{inputenc}
\usepackage[T1]{fontenc}    

\usepackage{amsmath}
\usepackage{amsfonts}
\usepackage{amssymb}
\usepackage{amsthm}
\usepackage{mathrsfs}

\usepackage{mathtools}
\usepackage{hyperref}
\usepackage{enumerate}

\usepackage{bbm} %% \mathbbm{1} gives you the identity function symbol 1

%%% allows you to insert many figures indexed by (a), (b), ... on
%%% a figure environment
\usepackage{float} 
\usepackage[caption = false]{subfig}

\usepackage{tikz}
\usetikzlibrary{snakes} %% produces curly arrows on tikz
\usetikzlibrary{matrix} %% for commutative diagrams
\usetikzlibrary{arrows}

\usepackage{ifthen} %% gives if conditionals when using newcommand

%% Garamond fonts 
%% \usepackage{ebgaramond}
%% \usepackage[ugm]{mathdesign}
% 1}}}

%|--- Math operators {{{1
\DeclareTextFontCommand {\emph}{\bfseries\em} %%\emph{} in bold and italic


\DeclareMathOperator {\R}{\mathbb{R}}
\DeclareMathOperator {\Q}{\mathbb{Q}}
\DeclareMathOperator {\N}{\mathbb{N}}
\DeclareMathOperator {\Z}{\mathbb{Z}}
\DeclareMathOperator {\card}{card}
\DeclareMathOperator {\Id}{\mathbbm{1}}
\DeclareMathOperator {\definedAs}{\vcentcolon = }
\DeclareMathOperator {\Imagem}{ Im }
% 1}}}

%|--- Theorems {{{1
\theoremstyle{remark}
\newtheorem{example}{Example}[section]

\theoremstyle{definition}
\newtheorem{observation}{Observation}[section]
\newtheorem{definition}{Definition}[section]
\newtheorem*{notation}{Notation}

\theoremstyle{plain}
\newtheorem{theorem}{Theorem}[section]
\newtheorem{proposition}{Proposition}[section]
% 1}}}





%|--- Author, Title and Date {{{1
\title{Exercise 1 - chapter 4. Book: "Introdução a Medida e Integração"} 
\author{Rafael Polli Carneiro}
\date{August 2021} 
% 1}}}

%|--- Begin Document {{{1
\begin{document}
\maketitle
\tableofcontents

Here we will prove the following statement
\begin{proposition}
    Let $(\Omega, \mathcal{A}, \lambda)$  be a measurable space where 
    $\lambda: \mathcal{A} \to [0,\infty]$  is the Lebesgue extension of the
    measure $\lambda: \mathcal{S} \to [0,\infty)$,
    $\mathcal{S}$ is the semi-ring of the limited intervals on $\R$, and
    \begin{equation*}
        \lambda(I) = \sup(I) - \inf(I), \forall I \in \mathcal{S}.
    \end{equation*}
    Now, for any $\alpha \in (0,1) \subseteq \R$ there is a measurable set
    $A_0$ whose interior is empty but its measurable is equal to $\alpha$:
    \begin{equation}
        \lambda(A_0) = \alpha.
    \end{equation}
\end{proposition}
\begin{proof}
   %\begin{comment}
   %Draft zone
   %\[
   %S_n = a_1 + a_2 + \ldots + a_n
   %aS_n = a_2 + a_3 + \ldots + a_{n+1} 
   %\implies
   %(1-a)S_n = (a_1 - a_{n+1}) \implies S_n = (a_1 - a_{n+1}) / (1-a)
   %\]
   %\end{comment}
   The idea follows the same steps done to create the ternary set of Cantor.
   Firstly, notice that, for any $\beta \in \R$ we have
   \begin{equation*}
    \sum_{i=1}^{\infty} \frac{\beta}{2^i}
    =
    \beta \sum_{i=1}^{\infty} \frac{1}{2^i}
    =
    \beta. 
   \end{equation*}
   This series gives us the steps necessary to produce each interval that
   should be taken away from the interval $[0,1]$. We will do it in the 
   following manner. For $i=1$ we define  
   \begin{equation*}
    K_1 = \bigcup_{a \in \gamma_1} \left[a, a+ \frac{1+\alpha}{2^{1+1}}\right],
   \end{equation*}
   with $\gamma_1 = \{0, \frac{1+\alpha}{2^{1+1}} + \frac{1-\alpha}{2^{1}}\}$,
   then, by induction, considering $K_n, \gamma_n$ well defined for all
   naturals, we set
   \begin{equation*}
    K_{n+1}= \bigcup_{a \in \gamma_{n+1}}
        \left[a, a+ \frac{1+\alpha}{2^{(n+1)+1}}\right],
   \end{equation*}
   with
   \begin{equation*}
    \gamma_{n+1} = 
    \left\{
        a + l(\frac{1+\alpha}{2^{2(n+1)}} + \frac{1-\alpha}{2^{2((n+1)-1)+1}});
        \; l = 0 \text{ or } l = 1 \text{ and } a \in \gamma_n 
    \right\}.
   \end{equation*}
    This is the same done at the construction of the Cantor's set, 
    differing only at the lenght of the slices being cut at each
    interaction. 

    It is worth mentioning that $1/4$ of each interval cut at interaction
    $n$  is less than the lenght of the intervals left at the same step
    of the iteraction. That is
    \begin{equation*}
        \frac{1-\alpha}{2^{2(n-1)+1}} 2^{-2} < 
        \frac{1+\alpha}{2^{2(n)}}.
    \end{equation*}

    Now, as the standard construction, we notice that for all 
    $n \in \N$  $K_n$ are closed sets (finite union of closed sets)
    and therefore the set
    \begin{equation*}
        K = \bigcap_{n \in \N} K_n
    \end{equation*}
    is closed as well. Since we are working with the $\sigma-$algebra
    of the borelian sets. We conclude that $K$ is a measurable set.
    Its interior is empty. In fact, since we are shrinking down the
    lenght of each interval within $K_n$  to zero, none interval can
    be included at $K$.  
    
    Finally, recall that at each step of the induction we remove
    $2^{n-1}$ disjoint intervals of length equal to
    $\frac{1-\alpha}{2^{2(n-1)+1}}$.  Thus
    \begin{equation*}
    \lambda(K) = 1 - \sum_{n=1}^{\infty} 2^{n-1}\frac{1-\alpha}{2^{2(n-1)+1}}
         = 1 - (1-\alpha)\sum_{n=1}^{\infty} 2^{-n}
         = \alpha
    \end{equation*}

\end{proof}
\end{document}
% 1}}}
