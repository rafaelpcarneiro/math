% vim: foldmethod=marker: 
\documentclass[11pt,twoside,a4paper]{article}

% vim: foldmethod=marker: 

%|--- Packages and Configs {{{1
\usepackage[utf8]{inputenc}
\usepackage[T1]{fontenc}    

\usepackage{amsmath}
\usepackage{amsfonts}
\usepackage{amssymb}
\usepackage{amsthm}
\usepackage{mathrsfs}

\usepackage{mathtools}
\usepackage{hyperref}
\usepackage{enumerate}

\usepackage{bbm} %% \mathbbm{1} gives you the identity function symbol 1

%%% allows you to insert many figures indexed by (a), (b), ... on
%%% a figure environment
\usepackage{float} 
\usepackage[caption = false]{subfig}

\usepackage{tikz}
\usetikzlibrary{snakes} %% produces curly arrows on tikz
\usetikzlibrary{matrix} %% for commutative diagrams
\usetikzlibrary{arrows}

\usepackage{ifthen} %% gives if conditionals when using newcommand

%% Garamond fonts 
%% \usepackage{ebgaramond}
%% \usepackage[ugm]{mathdesign}
% 1}}}

%|--- Math operators {{{1
\DeclareTextFontCommand {\emph}{\bfseries\em} %%\emph{} in bold and italic


\DeclareMathOperator {\R}{\mathbb{R}}
\DeclareMathOperator {\Q}{\mathbb{Q}}
\DeclareMathOperator {\N}{\mathbb{N}}
\DeclareMathOperator {\Z}{\mathbb{Z}}
\DeclareMathOperator {\card}{card}
\DeclareMathOperator {\Id}{\mathbbm{1}}
\DeclareMathOperator {\definedAs}{\vcentcolon = }
\DeclareMathOperator {\Imagem}{ Im }
% 1}}}

%|--- Theorems {{{1
\theoremstyle{remark}
\newtheorem{example}{Example}[section]

\theoremstyle{definition}
\newtheorem{observation}{Observation}[section]
\newtheorem{definition}{Definition}[section]
\newtheorem*{notation}{Notation}

\theoremstyle{plain}
\newtheorem{theorem}{Theorem}[section]
\newtheorem{proposition}{Proposition}[section]
% 1}}}




%|--- Author, Title and Date {{{1
\title{Exercise 7 - chapter 4. Book: "Introdução a Medida e Integração"} 
\author{Rafael Polli Carneiro}
\date{August 2021} 
% 1}}}

%|--- Begin Document {{{1
\begin{document}
\maketitle

\begin{definition}
    Let $(\Omega, \mathcal{A}, \mu)$ be a measurable space. Then, we say that 
    any measurable set $A \in \mathcal{A}$ is an atom if
    \begin{equation*}
        \forall B \in \mathcal{A}
        \left( 
            B \subseteq A \implies \mu(B) = 0 \text{ or } \mu(B) = \mu(A).
        \right)
    \end{equation*}
    Thus we say that a masure $\mu$ is non-atomic if it doesn't admit an
    atomic measurable space as element of its $\sigma-$algebra. In other
    words, $\mu$ is non-atomic if
    \begin{equation*}
        \forall A \in \mathcal{A},
        \exists B \in \mathcal{A}\setminus\{\emptyset\}
        \left(
        B \subseteq A \text{ and } 0 < \mu(B) < \mu(A)
        \right)
    \end{equation*}
\end{definition}

Now that we know the meaning of a measure being non-atomic we will proove
the following
\begin{proposition}
    Let $(\Omega, \mathcal{A},\mu)$ be a measurable space and
    $\mu$ a non-atomic measure. Then for any measurable set $A \in \mathcal{A}$    
    such that
    \begin{equation*}
        \mu(A)  < \infty
    \end{equation*}
    we have guaranteed that for any real number $\alpha \in \R$
    satisfying
    \begin{equation*}
        0 < \alpha < \mu(A)        
    \end{equation*}
    there is $B in \mathcal{A}$ where the equality
    \begin{equation*}
        \mu(B)  = \alpha
    \end{equation*}
    holds.
\end{proposition}
\begin{proof}
    The proof for this proposition is quite trick. I will make use of the
    Zorn's Lemma to proceed with the proof 
    (check the section below \ref{sec:zorn}).
\end{proof}

\section {Zorn's Lemma} \label{sec:zorn}
We say that a non empty set $X$ is endowed with a partial order $\leq$    
if
\begin{enumerate}[(i)]
    \item $\leq$ is \textit{reflexive}, which means
    \begin{equation*}
        \forall x \in X ( x \leq x)
    \end{equation*}

    \item $\leq$ is \textit{anti-simmetric}:
    \begin{equation*}
        \forall x,y \in X ( x \leq y, y \leq x \implies x = y )
    \end{equation*}
    \item $\leq$ is \textit{transitive}:
    \begin{equation*}
        \forall x,y,z \in X (x \leq y, y \leq z \implies x \leq z).
    \end{equation*}
\end{enumerate}
Every set $X$ in posse of a partial order is called as a partially ordered   
set. 

Before we adress the Zorn's Lemma we must introduction some notions.
Consider $X$ a partially ordered set and $Y \subseteq X$ a subset.
We say that 
\begin{enumerate}
    \item $x_0 \in X$ is an \textit{upper bound} of $Y$ if
    $\forall y \in Y, y \leq x_0$;

    \item $x_0 \in X$ is a \textit{maximal element} of $X$ if    
    $\forall x \in X, x \leq x_0$;

    \item $Y$ is a \textit{chain} if 
    $\forall x,y \in Y, x \leq y \text{ or } y \leq x$.
\end{enumerate}

Now we are able of stating the Zorn's Lemma, which is a proposition
equivalent to the axiom of choice
\begin{proposition}
    Let $X, \leq$ be a partially ordered set. If any chain
    $Y \subseteq X$ admits an upper bound on $X$, that is
    \begin{equation*}
        \forall Y \subseteq X, \forall y \in Y, \exists y_0 \in X
        \left(
          Y \text{ non empty and it is a chain } \implies y \leq y_0
        \right),
    \end{equation*}
    then $X$ does have a maximal element.  
\end{proposition}

\section{Inducing a partial order into a measurable space}
As usual, consider $(\Omega, \mathcal{A}, \mu)$ a measurable space.
($\Omega$ non empty). We say that two measurable sets $X,Y \in \mathcal{A}$  
are almost equal if 
\begin{equation*}
    \mu( X \setminus Y \cup Y \setminus X ) = 0.
\end{equation*}
From this definition we state the following relation
\begin{equation*}
    \forall X, Y \in \mathcal{A}
    X \sim Y \iff 
    \mu( X \setminus Y \cup Y \setminus X ) = 0.
\end{equation*}
Notice that
\begin{enumerate}[(i)]
   \item $\forall X \in \mathcal{A} \implies X \sim X$;
   
   \item $\forall X, Y \in \mathcal{A}$, clearly $X \sim Y$ 
   and $Y \sim X$;

   \item $\forall X, Y, Z \in \mathcal{A}$ such that
   \begin{equation*}
        X \sim Y \quad \text{and} \quad Y \sim Z
   \end{equation*}
   then 
   \begin{align*}
    \mu( Z \setminus X \cup X \setminus Z )
        & = \mu( [(Z \setminus X) \cup (X \setminus Z)] \cap (Y \cup Y^c) )\\
        & = \mu( [(Z \setminus X)\cap (Y \cup Y^c)]
               \cup 
               [(X \setminus Z) \cap (Y \cup Y^c)] )\\
        & = \mu( [(Z \cap X^c \cap Y) \cup (Z \cap X^c \cap Y^c)]
               \cup 
               [(X \cap Z^c \cap Y)  \cup (X \cap Z^c \cap Y^c)] )\\
        & \leq
            \mu( Z \cap X^c \cap Y)
            + \mu(Z \cap X^c \cap Y^c)
            + \mu(X \cap Z^c \cap Y) 
            + \mu(X \cap Z^c \cap Y^c)\\
        & \leq
            \mu(X^c \cap Y)
            + \mu(Z \cap Y^c)
            + \mu(Z^c \cap Y) 
            + \mu(X \cap Y^c)\\
        & = 0.
   \end{align*}
\end{enumerate}
Consequently, we conclude that $\sim$ is an equivalence relation and
therefore we can reduce $\mathcal{A}$ into its equivalence classes  
\begin{equation*}
    \mathcal{A} / \sim = \{ [A]; \; \forall A \in \mathcal{A} \}
\end{equation*}
where
\begin{equation*}
    \forall A \in \mathcal{A}, [A] = \{ B \in \mathcal{A}; \; B \sim A \}.
\end{equation*}

Now we are up to define a partial order over the sets on $\mathcal{A}/\sim$.  
Let $A, B \in \mathcal{A}/\sim$ be any sets, then we define
\begin{equation*}
    [A] \preceq [B] \iff \mu (A \setminus B ) = 0.
\end{equation*}

First we need to inspect if what is written above is well defined. 
That is, it doens't depend on the representations of the classes.
But first lets just define a notation which will decrease the notation.
Fo any sets $A,B$ we define
\begin{equation*}
    A \Delta B = (A \setminus B) \cup (B \setminus A).
\end{equation*}

Now we will show that $\preceq$ is well define.  
Fix sets satisfying
\begin{equation*}
    A_0 \sim A_1 \quad \text{and} \quad B_0 \sim B_1.
\end{equation*}
Then,
\begin{align*}
    \mu(B_1 \setminus A_1)
        & =
        \mu(B_1 \cap A_1^c) \\
        & =
        \mu(B_1 \cap (B_0 \cup B_0^c) \cap A_1^c) \\
        & =
        \mu(B_1 \cap B_0 \cap A_1^c) + 
        \mu(B_1 \cap B_0^c \cap A_1^c) \\
        & =
        \mu(B_1 \cap B_0 \cap A_1^c) \\
        & =
        \mu(B_1 \cap B_0 \cap A_1^c) +
        \mu(B_1^c \cap B_0 \cap A_1^c)   \\
        & =
        \mu(B_0 \cap A_1^c)  \\
        & =
        \mu(B_0 \cap A_1^c \cap (A_0 \cup A_0^c))  \\
        & =
        \mu(B_0 \cap A_1^c \cap A_0^c) \\
        & =
        \mu(B_0 \cap A_1^c \cap A_0^c) +
        \mu(B_0 \cap A_1  \cap A_0^c)  \\
        & =
        \mu(B_0 \cap A_0^c).\\
\end{align*}
Hence, what we conlude is
\begin{equation*}
    0 = \mu(B_0 \setminus A_0) \iff 0 = \mu(B_1 \setminus A_1)
\end{equation*}
which means $\preceq$ is well defined.  

Now we need to show that $\preceq$ is indeed a partial order.
Consider $A,B,C$ any measurable set. Thus
\begin{enumerate}[(i)]
    \item $[A] \preceq [A]$ (immediate to check); 

    \item $[A] \preceq [B]$ and $[B] \preceq [A]$  implies that  
    \begin{equation*}
        \mu(A \setminus B) = \mu(B \setminus A) = 0
    \end{equation*}
    which pretty much meas $A \sim B \implies [A] = [B]$.  

    \item $[A] \preceq [B]$ and $[B] \preceq [C]$ provide us with
    \begin{align*}
        \mu(A \setminus C)
            &=
            \mu(A \cap C^c \cap (B \cup B^c)) \\
            &=
            \mu(A \cap C^c \cap B) +
            \mu(A \cap C^c \cap B^c) \\
            & \leq
            \mu(C^c \cap B) + \mu(A  \cap B^c) \\
            & = 0.
    \end{align*}
    Consequently, $[A] \preceq [C]$  
\end{enumerate}

After all this we have just concluded that 
$(\mathcal{A}/\sim, \preceq)$ is a partially ordered set.
An important property from the equivalence class is:
\begin{equation*}
    \forall A \in \mathcal{A}, \forall B,C \in \mathcal{A}/\sim
    \Big( 
        \mu(B) = \mu(C)
    \Big).
\end{equation*}
\end{document}
% 1}}}
