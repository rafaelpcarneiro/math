% vim: foldmethod=marker: 

%|--- Packages and Configs {{{1
\usepackage[utf8]{inputenc}
\usepackage[T1]{fontenc}    

\usepackage{amsmath}
\usepackage{amsfonts}
\usepackage{amssymb}
\usepackage{amsthm}
\usepackage{mathrsfs}

\usepackage{mathtools}
\usepackage{hyperref}
\usepackage{enumerate}

\usepackage{bbm} %% \mathbbm{1} gives you the identity function symbol 1

%%% allows you to insert many figures indexed by (a), (b), ... on
%%% a figure environment
\usepackage{float} 
\usepackage[caption = false]{subfig}

\usepackage{tikz}
\usetikzlibrary{snakes} %% produces curly arrows on tikz
\usetikzlibrary{matrix} %% for commutative diagrams
\usetikzlibrary{arrows}

\usepackage{ifthen} %% gives if conditionals when using newcommand

%% Garamond fonts 
%% \usepackage{ebgaramond}
%% \usepackage[ugm]{mathdesign}
% 1}}}

%|--- Math operators {{{1
\DeclareTextFontCommand {\emph}{\bfseries\em} %%\emph{} in bold and italic


\DeclareMathOperator {\R}{\mathbb{R}}
\DeclareMathOperator {\Q}{\mathbb{Q}}
\DeclareMathOperator {\N}{\mathbb{N}}
\DeclareMathOperator {\Z}{\mathbb{Z}}
\DeclareMathOperator {\card}{card}
\DeclareMathOperator {\Id}{\mathbbm{1}}
\DeclareMathOperator {\definedAs}{\vcentcolon = }
\DeclareMathOperator {\Imagem}{ Im }
% 1}}}

%|--- Theorems {{{1
\theoremstyle{remark}
\newtheorem{example}{Example}[section]

\theoremstyle{definition}
\newtheorem{observation}{Observation}[section]
\newtheorem{definition}{Definition}[section]
\newtheorem*{notation}{Notation}

\theoremstyle{plain}
\newtheorem{theorem}{Theorem}[section]
\newtheorem{proposition}{Proposition}[section]
% 1}}}


